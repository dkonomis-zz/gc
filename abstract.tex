%% Greek Abstract
\begin{greek}
\begin{abstract}

Αν και αρχικά περιοριζόταν στη γλώσσα Lisp και τις
συναρτησιακές γλώσσες προγραμματισμού, σήμερα η συλλογή σκουπιδιών αποτελεί
σημαντικό κομμάτι του συστήματος διαχείρισης μνήμης πολλών σύγχρονων γλωσσών,
τόσο προστακτικών όσο και δηλωτικών.  Παρά τη φήμη πως
επιβραδύνει και διαταράσσει την εκτέλεση διαδραστικών εφαρμογών, οι σύγχρονες
 τεχνικές υλοποίησης της συλλογής σκουπιδιών έχουν μειώσει την επιβάρυνση
σημαντικά, σε τέτοιο σημείο ώστε να αποτελεί
ρεαλιστική επιλογή ακόμη και για παραδοσιακές γλώσσες προγραμματισμού όπως η C.

Παρά την ταχεία ανάπτυξη του μεγέθους μνήμης ακόμη και των πιο φθηνών
υπολογιστών, ο αποθηκευτικός χώρος δεν είναι ανεξάντλητος. Όπως όλοι οι
περιορισμένοι πόροι απαιτεί προσεκτική συντήρηση και ανακύκλωση. Πολλές γλώσσες
προγραμματισμού σήμερα επιτρέπουν την εκχώρηση και την ανάκτηση μνήμης από τον
προγραμματιστή για δεδομένα των οποίων η διάρκεια ζωής δεν καθορίζεται από
την λεκτική εμβέλεια τους. Λέμε πώς τα δεδομένα αυτά
\emph{εκχωρούνται δυναμικά}. Η διαχείριση της δυναμικής μνήμης μπορεί να γίνεται
ρητώς από
τον προγραμματιστή, μέσω κλήσεων διαδικασιών ενσωματωμένων στο σύστημα εκτέλεσης
ή διαδικασιών βιβλιοθήκης που εκχωρούν αποθηκευτικό χώρο και τον ελευθερώνουν
όταν αυτός δεν είναι πλέον απαραίτητος.

Η χειρωνακτική ανάκτηση της δυναμικά διαχειριζόμενης μνήμης συχνά δεν είναι
ικανοποιητική. Εναλλακτικά, η ευθύνη αυτής της διαχείρισης
μπορεί να μεταβιβασθεί στο σύστημα εκτέλεσης του προγράμματος. Ο
προγραμματιστής εξακολουθεί να πραγματοποιεί δυναμικά αιτήματα εκχώρησης
μνήμης, χωρίς όμως πλέον να χρειάζεται να καθορίσει πότε αυτή η μνήμη δεν είναι
πλέον απαραίτητη: ανακυκλώνεται αυτόματα. H \emph{συλλογή σκουπιδιών},
που αποτελεί το κεντρικό θέμα αυτής της εργασίας, είναι ακριβώς η αυτόματη
διαχείριση δυναμικά εκχωρούμενου χώρου αποθήκευσης.

Το εισαγωγικό κεφάλαιο αυτής της εργασίας αρχικά εξετάζει και συγκρίνει την
αυτόματη διαχείριση μνήμης και τη ρητή διαχείριση μνήμης, στη συνέχεια
παρουσιάζει τις μετρικές με βάση τις οποίες συγκρίνονται τα διάφορα σχήματα
συλλογής σκουπιδιών και τέλος ορίζει τις έννοιες του \emph{εκχωρητή}, του
\emph{συλλέκτη} και του \emph{τροποποιητή}. Το υπόλοιπο της εργασίας οργανώνεται
σε δύο τμήματα.  Στο πρώτο συζητώνται οι κλασσικές προσεγγίσεις
συλλογής σκουπιδιών: \emph{σήμανση και εκκαθάριση},
\emph{σήμανση και συμπύκνωση}, \emph{αντιγραφή} και
\emph{καταμέτρηση αναφορών}. Το δεύτερο μέρος επικεντρώνεται σε πιο προηγμένες
τεχνικές συλλογής σκουπιδιών: \emph{γενεαλογική}, \emph{παράλληλη} και
\emph{ταυτόχρονη} συλλογή σκουπιδιών. Τέλος εισάγουμε τη συλλογή σκουπιδιών
\emph{πραγματικού χρόνου}.

Η εργασία αυτή βασίζεται σε μετάφραση και προσαρμογή στα ελληνικά τμημάτων
του βιβλίου \textit{The Garbage Collection Handbook: The Art of Automatic
  Memory Management}, των R.~Jones, A.~Hoskins και E.~Moss
 \cite{DBLP:books/wi/Jones2011}.

\begin{keywords}
γλώσσες προγραμματισμού, αυτόματη διαχείριση μνήμης, συλλογή σκουπιδιών,
σήμανση και εκκαθάριση, σήμανση και συμπύκνωση, αντιγραφή, καταμέτρηση αναφορών,
γενεαλογική, παράλληλη, ταυτόχρονη, πραγματικού χρόνου.
\end{keywords}

\end{abstract}
\end{greek}

%% English Abstract
\begin{abstract}
Whereas it was once confined to the realm of Lisp and functional languages, today
garbage collection is an important part of the memory management system of many
modern programming languages, imperative as well as declarative. Although
garbage collection has had a reputation for sloth and for disrupting interactive
programs, modern implementation techniques have reduced its overheads
substantially, to the point where garbage collected heaps are a realistic option
even for traditional languages like C.

Despite the rapid growth in memory sizes of even the most modest computers, the
supply of storage is not inexhaustible. Like all limited resources it requires
careful conservation and recycling. Many programming languages today allow the
programmer to allocate and reclaim memory for data whose lifetimes are not
determined by lexical scope. Such data is said to be
\emph{dynamically allocated}. Dynamic memory may be managed explicitly by the
programmer through invocations of built-in or library procedures that allocate
storage and that dispose or free that storage when it is no longer needed.

Manual reclamation of dynamically managed storage is often unsatisfactory. The
alternative is to devolve responsibility for dynamic memory management to the
program's run-time system. The programmer must still request dynamically
allocated storage to be reserved but no longer needs to determine when that
memory is no longer required: it is recycled automatically. Garbage collection,
the main topic of this thesis, is precisely this: the automatic management of
dynamically allocated storage.

The introductory chapter of this thesis first examines and compares automatic
memory management and explicit memory management, then presents the metrics
under which different garbage collection schemes are compared and finally
defines the notion of the \emph{allocator}, \emph{collector} and \emph{mutator}.
The rest of the thesis is organized as follows: the first part discusses the
classical garbage collection approaches. We examine \emph{mark-sweep},
\emph{mark-compact}, \emph{copying} and \emph{reference counting}. The second
part focuses on more advanced garbage collection techniques. We examine
\emph{generational}, \emph{parallel} and \emph{concurrent} garbage collection.
Finally, we introduce \emph{real-time} garbage collection.

This thesis is based on the translation and adaptation to Greek of parts
of the book \textit{The Garbage Collection Handbook: The Art of Automatic
  Memory Management}, by R.~Jones, A.~Hoskins and E.~Moss
 \cite{DBLP:books/wi/Jones2011}.

\begin{keywordseng}
programming languages, automatic memory management, garbage collection,
mark-sweep, mark-compact, copying, reference counting, generational, parallel,
concurrent, real-time.
\end{keywordseng}

\end{abstract}

%%Greek Acknowledgements
\begin{greek}
\begin{acknowledgements}
Με την παρούσα διπλωματική εργασία ολοκληρώνονται οι σπουδές μου στην Σχολή 
Ηλεκτρολόγων Μηχανικών \& Μηχανικών Υπολογιστών του Εθνικού Μετσοβίου 
Πολυτεχνείου.

Αισθάνομαι τη βαθύτατη ανάγκη να ευχαριστήσω τον επιβλέποντα αυτής της εργασίας 
καθηγητή Νίκο Παπασπύρου, ο οποίος εκτός από εξαίρετος επιστήμονας και δάσκαλος, 
είναι και σπουδαίος άνθρωπος. Είναι εκείνος που συνέβαλλε στο να αγαπήσω την 
Πληροφορική και τις Γλώσσες Προγραμματισμού ειδικότερα. 

Παράλληλα θέλω να ευχαριστήσω θερμά και τους καθηγητές μου Κωστή Σαγώνα, 
Κώστα Κοντογιάννη, Δημήτρη Φωτάκη, Τίμο Σελλή και Στάθη Ζάχο από τους οποίους
επίσης διδάχθηκα πάρα πολλά.

Κατά τη διάρκεια των σπουδών μου όμως δεν έμαθα μόνο από τους καθηγητές μου, αλλά 
και από τους φίλους μου. Ευχαριστώ τον Ηλία, το Ζήση, το Στέφανο, το Νίκο, 
τον Κωνσταντίνο, το Νίκο, το Διονύση και το Βρεττό.

Last και σίγουρα not least, οφείλω ένα μεγάλο ευχαριστώ στην οικογένειά μου, που, 
παρά τις όποιες δυσκολίες, με στήριξε και βοήθησε στο να φτάσω εδώ που 
είμαι σήμερα.
\begin{flushright}Δημήτρης X. Κονόμης\end{flushright}

\vfill

    Η εργασία αυτή είναι επίσης διαθέσιμη ως Τεχνική Αναφορά
    CSD-TR-1-15, Εθνικό Μετσόβιο Πολυτεχνείο, Σχολή Ηλεκτρολόγων
    Μηχανικών και Μηχανικών Υπολογιστών, Τομέας Τεχνολογίας
    Πληροφορικής και Υπολογιστών, Εργαστήριο Τεχνολογίας
    Λογισμικού, Απρίλιος 2015.
    \\[6pt]
    \begin{tabular}{@{}ll@{}}
    URL: & \texttt{http://www.softlab.ntua.gr/techrep/} \\
    FTP: & \texttt{ftp://ftp.softlab.ntua.gr/pub/techrep/}
    \end{tabular}

\end{acknowledgements}
\end{greek}
