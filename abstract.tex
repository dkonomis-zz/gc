%% Greek Abstract
\begin{greek}
\begin{abstract}



\begin{keywords}

\end{keywords}

\end{abstract}
\end{greek}


%% English Abstract
\begin{abstract}

Automatic memory management 

(?? SOMETHING TO SAY ITS ACTIVE AND INTERESTING ??)

All garbage collection schemes are based on one of four
fundamental approaches: \emph{mark-sweep collection},
\emph{copying collection}, \emph{mark-compact collection}
or \emph{reference counting}. Different collectors may
combine these approaches in different ways.

Generational collectors for example segregate objects by
age into \emph{generations}, typically physically distinct
areas of the heap. Younger generations are collected in
preference to older ones, and objects that survive long
enough are promoted from the generation being collected
to an older one. 




κ
\begin{keywordseng}

\end{keywordseng}

\end{abstract}


%%Greek Acknowledgements
\begin{greek}
\begin{acknowledgements}
Με την παρούσα διπλωματική εργασία ολοκληρώνονται οι σπουδές μου στην Σχολή 
Ηλεκτρολόγων Μηχανικών \& Μηχανικών Υπολογιστών του Εθνικού Μετσοβίου 
Πολυτεχνείου.

Αισθάνομαι τη βαθύτατη ανάγκη να ευχαριστήσω τον επιβλέποντα αυτής της εργασίας 
καθηγητή Νίκο Παπασπύρου, ο οποίος εκτός από εξαίρετος επιστήμονας και δάσκαλος, 
είναι και σπουδαίος άνθρωπος. Είναι εκείνος που συνέβαλλε στο να αγαπήσω την 
Πληροφορική και τις Γλώσσες Προγραμματισμού ειδικότερα. 

Παράλληλα θέλω να ευχαριστήσω θερμά και τους καθηγητές μου Κωστή Σαγώνα, 
Στάθη Ζάχο, Δημήτρη Φωτάκη, Τίμο Σελλή και Κώστα Κοντογιάννη από τους οποίους επίσης διδάχθηκα 
πάρα πολλά.

Κατά τη διάρκεια των σπουδών μου όμως δεν έμαθα μόνο από τους καθηγητές μου, αλλά 
και από τους φίλους μου. Ευχαριστώ τον Ηλία, το Ζήση, το Στέφανο, το Νίκο, 
τον Κωνσταντίνο, το Νίκο, το Διονύση και το Βρεττό.

Last και σίγουρα not least, οφείλω ένα μεγάλο ευχαριστώ στην οικογένειά μου, που, 
παρά τις όποιες δυσκολίες, με στήριξε και βοήθησε στο να φτάσω εδώ που 
είμαι σήμερα.
\begin{flushright}Δημήτρης X. Κονόμης\end{flushright}
\end{acknowledgements}
\end{greek}
